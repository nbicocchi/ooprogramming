
% Default to the notebook output style

    


% Inherit from the specified cell style.




    
\documentclass[11pt]{article}

    
    
    \usepackage[T1]{fontenc}
    % Nicer default font (+ math font) than Computer Modern for most use cases
    \usepackage{mathpazo}

    % Basic figure setup, for now with no caption control since it's done
    % automatically by Pandoc (which extracts ![](path) syntax from Markdown).
    \usepackage{graphicx}
    % We will generate all images so they have a width \maxwidth. This means
    % that they will get their normal width if they fit onto the page, but
    % are scaled down if they would overflow the margins.
    \makeatletter
    \def\maxwidth{\ifdim\Gin@nat@width>\linewidth\linewidth
    \else\Gin@nat@width\fi}
    \makeatother
    \let\Oldincludegraphics\includegraphics
    % Set max figure width to be 80% of text width, for now hardcoded.
    \renewcommand{\includegraphics}[1]{\Oldincludegraphics[width=.8\maxwidth]{#1}}
    % Ensure that by default, figures have no caption (until we provide a
    % proper Figure object with a Caption API and a way to capture that
    % in the conversion process - todo).
    \usepackage{caption}
    \DeclareCaptionLabelFormat{nolabel}{}
    \captionsetup{labelformat=nolabel}

    \usepackage{adjustbox} % Used to constrain images to a maximum size 
    \usepackage{xcolor} % Allow colors to be defined
    \usepackage{enumerate} % Needed for markdown enumerations to work
    \usepackage{geometry} % Used to adjust the document margins
    \usepackage{amsmath} % Equations
    \usepackage{amssymb} % Equations
    \usepackage{textcomp} % defines textquotesingle
    % Hack from http://tex.stackexchange.com/a/47451/13684:
    \AtBeginDocument{%
        \def\PYZsq{\textquotesingle}% Upright quotes in Pygmentized code
    }
    \usepackage{upquote} % Upright quotes for verbatim code
    \usepackage{eurosym} % defines \euro
    \usepackage[mathletters]{ucs} % Extended unicode (utf-8) support
    \usepackage[utf8x]{inputenc} % Allow utf-8 characters in the tex document
    \usepackage{fancyvrb} % verbatim replacement that allows latex
    \usepackage{grffile} % extends the file name processing of package graphics 
                         % to support a larger range 
    % The hyperref package gives us a pdf with properly built
    % internal navigation ('pdf bookmarks' for the table of contents,
    % internal cross-reference links, web links for URLs, etc.)
    \usepackage{hyperref}
    \usepackage{longtable} % longtable support required by pandoc >1.10
    \usepackage{booktabs}  % table support for pandoc > 1.12.2
    \usepackage[inline]{enumitem} % IRkernel/repr support (it uses the enumerate* environment)
    \usepackage[normalem]{ulem} % ulem is needed to support strikethroughs (\sout)
                                % normalem makes italics be italics, not underlines
    

    
    
    % Colors for the hyperref package
    \definecolor{urlcolor}{rgb}{0,.145,.698}
    \definecolor{linkcolor}{rgb}{.71,0.21,0.01}
    \definecolor{citecolor}{rgb}{.12,.54,.11}

    % ANSI colors
    \definecolor{ansi-black}{HTML}{3E424D}
    \definecolor{ansi-black-intense}{HTML}{282C36}
    \definecolor{ansi-red}{HTML}{E75C58}
    \definecolor{ansi-red-intense}{HTML}{B22B31}
    \definecolor{ansi-green}{HTML}{00A250}
    \definecolor{ansi-green-intense}{HTML}{007427}
    \definecolor{ansi-yellow}{HTML}{DDB62B}
    \definecolor{ansi-yellow-intense}{HTML}{B27D12}
    \definecolor{ansi-blue}{HTML}{208FFB}
    \definecolor{ansi-blue-intense}{HTML}{0065CA}
    \definecolor{ansi-magenta}{HTML}{D160C4}
    \definecolor{ansi-magenta-intense}{HTML}{A03196}
    \definecolor{ansi-cyan}{HTML}{60C6C8}
    \definecolor{ansi-cyan-intense}{HTML}{258F8F}
    \definecolor{ansi-white}{HTML}{C5C1B4}
    \definecolor{ansi-white-intense}{HTML}{A1A6B2}

    % commands and environments needed by pandoc snippets
    % extracted from the output of `pandoc -s`
    \providecommand{\tightlist}{%
      \setlength{\itemsep}{0pt}\setlength{\parskip}{0pt}}
    \DefineVerbatimEnvironment{Highlighting}{Verbatim}{commandchars=\\\{\}}
    % Add ',fontsize=\small' for more characters per line
    \newenvironment{Shaded}{}{}
    \newcommand{\KeywordTok}[1]{\textcolor[rgb]{0.00,0.44,0.13}{\textbf{{#1}}}}
    \newcommand{\DataTypeTok}[1]{\textcolor[rgb]{0.56,0.13,0.00}{{#1}}}
    \newcommand{\DecValTok}[1]{\textcolor[rgb]{0.25,0.63,0.44}{{#1}}}
    \newcommand{\BaseNTok}[1]{\textcolor[rgb]{0.25,0.63,0.44}{{#1}}}
    \newcommand{\FloatTok}[1]{\textcolor[rgb]{0.25,0.63,0.44}{{#1}}}
    \newcommand{\CharTok}[1]{\textcolor[rgb]{0.25,0.44,0.63}{{#1}}}
    \newcommand{\StringTok}[1]{\textcolor[rgb]{0.25,0.44,0.63}{{#1}}}
    \newcommand{\CommentTok}[1]{\textcolor[rgb]{0.38,0.63,0.69}{\textit{{#1}}}}
    \newcommand{\OtherTok}[1]{\textcolor[rgb]{0.00,0.44,0.13}{{#1}}}
    \newcommand{\AlertTok}[1]{\textcolor[rgb]{1.00,0.00,0.00}{\textbf{{#1}}}}
    \newcommand{\FunctionTok}[1]{\textcolor[rgb]{0.02,0.16,0.49}{{#1}}}
    \newcommand{\RegionMarkerTok}[1]{{#1}}
    \newcommand{\ErrorTok}[1]{\textcolor[rgb]{1.00,0.00,0.00}{\textbf{{#1}}}}
    \newcommand{\NormalTok}[1]{{#1}}
    
    % Additional commands for more recent versions of Pandoc
    \newcommand{\ConstantTok}[1]{\textcolor[rgb]{0.53,0.00,0.00}{{#1}}}
    \newcommand{\SpecialCharTok}[1]{\textcolor[rgb]{0.25,0.44,0.63}{{#1}}}
    \newcommand{\VerbatimStringTok}[1]{\textcolor[rgb]{0.25,0.44,0.63}{{#1}}}
    \newcommand{\SpecialStringTok}[1]{\textcolor[rgb]{0.73,0.40,0.53}{{#1}}}
    \newcommand{\ImportTok}[1]{{#1}}
    \newcommand{\DocumentationTok}[1]{\textcolor[rgb]{0.73,0.13,0.13}{\textit{{#1}}}}
    \newcommand{\AnnotationTok}[1]{\textcolor[rgb]{0.38,0.63,0.69}{\textbf{\textit{{#1}}}}}
    \newcommand{\CommentVarTok}[1]{\textcolor[rgb]{0.38,0.63,0.69}{\textbf{\textit{{#1}}}}}
    \newcommand{\VariableTok}[1]{\textcolor[rgb]{0.10,0.09,0.49}{{#1}}}
    \newcommand{\ControlFlowTok}[1]{\textcolor[rgb]{0.00,0.44,0.13}{\textbf{{#1}}}}
    \newcommand{\OperatorTok}[1]{\textcolor[rgb]{0.40,0.40,0.40}{{#1}}}
    \newcommand{\BuiltInTok}[1]{{#1}}
    \newcommand{\ExtensionTok}[1]{{#1}}
    \newcommand{\PreprocessorTok}[1]{\textcolor[rgb]{0.74,0.48,0.00}{{#1}}}
    \newcommand{\AttributeTok}[1]{\textcolor[rgb]{0.49,0.56,0.16}{{#1}}}
    \newcommand{\InformationTok}[1]{\textcolor[rgb]{0.38,0.63,0.69}{\textbf{\textit{{#1}}}}}
    \newcommand{\WarningTok}[1]{\textcolor[rgb]{0.38,0.63,0.69}{\textbf{\textit{{#1}}}}}
    
    
    % Define a nice break command that doesn't care if a line doesn't already
    % exist.
    \def\br{\hspace*{\fill} \\* }
    % Math Jax compatability definitions
    \def\gt{>}
    \def\lt{<}
    % Document parameters
    \title{04 - Scientific Python}
    
    
    

    % Pygments definitions
    
\makeatletter
\def\PY@reset{\let\PY@it=\relax \let\PY@bf=\relax%
    \let\PY@ul=\relax \let\PY@tc=\relax%
    \let\PY@bc=\relax \let\PY@ff=\relax}
\def\PY@tok#1{\csname PY@tok@#1\endcsname}
\def\PY@toks#1+{\ifx\relax#1\empty\else%
    \PY@tok{#1}\expandafter\PY@toks\fi}
\def\PY@do#1{\PY@bc{\PY@tc{\PY@ul{%
    \PY@it{\PY@bf{\PY@ff{#1}}}}}}}
\def\PY#1#2{\PY@reset\PY@toks#1+\relax+\PY@do{#2}}

\expandafter\def\csname PY@tok@w\endcsname{\def\PY@tc##1{\textcolor[rgb]{0.73,0.73,0.73}{##1}}}
\expandafter\def\csname PY@tok@c\endcsname{\let\PY@it=\textit\def\PY@tc##1{\textcolor[rgb]{0.25,0.50,0.50}{##1}}}
\expandafter\def\csname PY@tok@cp\endcsname{\def\PY@tc##1{\textcolor[rgb]{0.74,0.48,0.00}{##1}}}
\expandafter\def\csname PY@tok@k\endcsname{\let\PY@bf=\textbf\def\PY@tc##1{\textcolor[rgb]{0.00,0.50,0.00}{##1}}}
\expandafter\def\csname PY@tok@kp\endcsname{\def\PY@tc##1{\textcolor[rgb]{0.00,0.50,0.00}{##1}}}
\expandafter\def\csname PY@tok@kt\endcsname{\def\PY@tc##1{\textcolor[rgb]{0.69,0.00,0.25}{##1}}}
\expandafter\def\csname PY@tok@o\endcsname{\def\PY@tc##1{\textcolor[rgb]{0.40,0.40,0.40}{##1}}}
\expandafter\def\csname PY@tok@ow\endcsname{\let\PY@bf=\textbf\def\PY@tc##1{\textcolor[rgb]{0.67,0.13,1.00}{##1}}}
\expandafter\def\csname PY@tok@nb\endcsname{\def\PY@tc##1{\textcolor[rgb]{0.00,0.50,0.00}{##1}}}
\expandafter\def\csname PY@tok@nf\endcsname{\def\PY@tc##1{\textcolor[rgb]{0.00,0.00,1.00}{##1}}}
\expandafter\def\csname PY@tok@nc\endcsname{\let\PY@bf=\textbf\def\PY@tc##1{\textcolor[rgb]{0.00,0.00,1.00}{##1}}}
\expandafter\def\csname PY@tok@nn\endcsname{\let\PY@bf=\textbf\def\PY@tc##1{\textcolor[rgb]{0.00,0.00,1.00}{##1}}}
\expandafter\def\csname PY@tok@ne\endcsname{\let\PY@bf=\textbf\def\PY@tc##1{\textcolor[rgb]{0.82,0.25,0.23}{##1}}}
\expandafter\def\csname PY@tok@nv\endcsname{\def\PY@tc##1{\textcolor[rgb]{0.10,0.09,0.49}{##1}}}
\expandafter\def\csname PY@tok@no\endcsname{\def\PY@tc##1{\textcolor[rgb]{0.53,0.00,0.00}{##1}}}
\expandafter\def\csname PY@tok@nl\endcsname{\def\PY@tc##1{\textcolor[rgb]{0.63,0.63,0.00}{##1}}}
\expandafter\def\csname PY@tok@ni\endcsname{\let\PY@bf=\textbf\def\PY@tc##1{\textcolor[rgb]{0.60,0.60,0.60}{##1}}}
\expandafter\def\csname PY@tok@na\endcsname{\def\PY@tc##1{\textcolor[rgb]{0.49,0.56,0.16}{##1}}}
\expandafter\def\csname PY@tok@nt\endcsname{\let\PY@bf=\textbf\def\PY@tc##1{\textcolor[rgb]{0.00,0.50,0.00}{##1}}}
\expandafter\def\csname PY@tok@nd\endcsname{\def\PY@tc##1{\textcolor[rgb]{0.67,0.13,1.00}{##1}}}
\expandafter\def\csname PY@tok@s\endcsname{\def\PY@tc##1{\textcolor[rgb]{0.73,0.13,0.13}{##1}}}
\expandafter\def\csname PY@tok@sd\endcsname{\let\PY@it=\textit\def\PY@tc##1{\textcolor[rgb]{0.73,0.13,0.13}{##1}}}
\expandafter\def\csname PY@tok@si\endcsname{\let\PY@bf=\textbf\def\PY@tc##1{\textcolor[rgb]{0.73,0.40,0.53}{##1}}}
\expandafter\def\csname PY@tok@se\endcsname{\let\PY@bf=\textbf\def\PY@tc##1{\textcolor[rgb]{0.73,0.40,0.13}{##1}}}
\expandafter\def\csname PY@tok@sr\endcsname{\def\PY@tc##1{\textcolor[rgb]{0.73,0.40,0.53}{##1}}}
\expandafter\def\csname PY@tok@ss\endcsname{\def\PY@tc##1{\textcolor[rgb]{0.10,0.09,0.49}{##1}}}
\expandafter\def\csname PY@tok@sx\endcsname{\def\PY@tc##1{\textcolor[rgb]{0.00,0.50,0.00}{##1}}}
\expandafter\def\csname PY@tok@m\endcsname{\def\PY@tc##1{\textcolor[rgb]{0.40,0.40,0.40}{##1}}}
\expandafter\def\csname PY@tok@gh\endcsname{\let\PY@bf=\textbf\def\PY@tc##1{\textcolor[rgb]{0.00,0.00,0.50}{##1}}}
\expandafter\def\csname PY@tok@gu\endcsname{\let\PY@bf=\textbf\def\PY@tc##1{\textcolor[rgb]{0.50,0.00,0.50}{##1}}}
\expandafter\def\csname PY@tok@gd\endcsname{\def\PY@tc##1{\textcolor[rgb]{0.63,0.00,0.00}{##1}}}
\expandafter\def\csname PY@tok@gi\endcsname{\def\PY@tc##1{\textcolor[rgb]{0.00,0.63,0.00}{##1}}}
\expandafter\def\csname PY@tok@gr\endcsname{\def\PY@tc##1{\textcolor[rgb]{1.00,0.00,0.00}{##1}}}
\expandafter\def\csname PY@tok@ge\endcsname{\let\PY@it=\textit}
\expandafter\def\csname PY@tok@gs\endcsname{\let\PY@bf=\textbf}
\expandafter\def\csname PY@tok@gp\endcsname{\let\PY@bf=\textbf\def\PY@tc##1{\textcolor[rgb]{0.00,0.00,0.50}{##1}}}
\expandafter\def\csname PY@tok@go\endcsname{\def\PY@tc##1{\textcolor[rgb]{0.53,0.53,0.53}{##1}}}
\expandafter\def\csname PY@tok@gt\endcsname{\def\PY@tc##1{\textcolor[rgb]{0.00,0.27,0.87}{##1}}}
\expandafter\def\csname PY@tok@err\endcsname{\def\PY@bc##1{\setlength{\fboxsep}{0pt}\fcolorbox[rgb]{1.00,0.00,0.00}{1,1,1}{\strut ##1}}}
\expandafter\def\csname PY@tok@kc\endcsname{\let\PY@bf=\textbf\def\PY@tc##1{\textcolor[rgb]{0.00,0.50,0.00}{##1}}}
\expandafter\def\csname PY@tok@kd\endcsname{\let\PY@bf=\textbf\def\PY@tc##1{\textcolor[rgb]{0.00,0.50,0.00}{##1}}}
\expandafter\def\csname PY@tok@kn\endcsname{\let\PY@bf=\textbf\def\PY@tc##1{\textcolor[rgb]{0.00,0.50,0.00}{##1}}}
\expandafter\def\csname PY@tok@kr\endcsname{\let\PY@bf=\textbf\def\PY@tc##1{\textcolor[rgb]{0.00,0.50,0.00}{##1}}}
\expandafter\def\csname PY@tok@bp\endcsname{\def\PY@tc##1{\textcolor[rgb]{0.00,0.50,0.00}{##1}}}
\expandafter\def\csname PY@tok@fm\endcsname{\def\PY@tc##1{\textcolor[rgb]{0.00,0.00,1.00}{##1}}}
\expandafter\def\csname PY@tok@vc\endcsname{\def\PY@tc##1{\textcolor[rgb]{0.10,0.09,0.49}{##1}}}
\expandafter\def\csname PY@tok@vg\endcsname{\def\PY@tc##1{\textcolor[rgb]{0.10,0.09,0.49}{##1}}}
\expandafter\def\csname PY@tok@vi\endcsname{\def\PY@tc##1{\textcolor[rgb]{0.10,0.09,0.49}{##1}}}
\expandafter\def\csname PY@tok@vm\endcsname{\def\PY@tc##1{\textcolor[rgb]{0.10,0.09,0.49}{##1}}}
\expandafter\def\csname PY@tok@sa\endcsname{\def\PY@tc##1{\textcolor[rgb]{0.73,0.13,0.13}{##1}}}
\expandafter\def\csname PY@tok@sb\endcsname{\def\PY@tc##1{\textcolor[rgb]{0.73,0.13,0.13}{##1}}}
\expandafter\def\csname PY@tok@sc\endcsname{\def\PY@tc##1{\textcolor[rgb]{0.73,0.13,0.13}{##1}}}
\expandafter\def\csname PY@tok@dl\endcsname{\def\PY@tc##1{\textcolor[rgb]{0.73,0.13,0.13}{##1}}}
\expandafter\def\csname PY@tok@s2\endcsname{\def\PY@tc##1{\textcolor[rgb]{0.73,0.13,0.13}{##1}}}
\expandafter\def\csname PY@tok@sh\endcsname{\def\PY@tc##1{\textcolor[rgb]{0.73,0.13,0.13}{##1}}}
\expandafter\def\csname PY@tok@s1\endcsname{\def\PY@tc##1{\textcolor[rgb]{0.73,0.13,0.13}{##1}}}
\expandafter\def\csname PY@tok@mb\endcsname{\def\PY@tc##1{\textcolor[rgb]{0.40,0.40,0.40}{##1}}}
\expandafter\def\csname PY@tok@mf\endcsname{\def\PY@tc##1{\textcolor[rgb]{0.40,0.40,0.40}{##1}}}
\expandafter\def\csname PY@tok@mh\endcsname{\def\PY@tc##1{\textcolor[rgb]{0.40,0.40,0.40}{##1}}}
\expandafter\def\csname PY@tok@mi\endcsname{\def\PY@tc##1{\textcolor[rgb]{0.40,0.40,0.40}{##1}}}
\expandafter\def\csname PY@tok@il\endcsname{\def\PY@tc##1{\textcolor[rgb]{0.40,0.40,0.40}{##1}}}
\expandafter\def\csname PY@tok@mo\endcsname{\def\PY@tc##1{\textcolor[rgb]{0.40,0.40,0.40}{##1}}}
\expandafter\def\csname PY@tok@ch\endcsname{\let\PY@it=\textit\def\PY@tc##1{\textcolor[rgb]{0.25,0.50,0.50}{##1}}}
\expandafter\def\csname PY@tok@cm\endcsname{\let\PY@it=\textit\def\PY@tc##1{\textcolor[rgb]{0.25,0.50,0.50}{##1}}}
\expandafter\def\csname PY@tok@cpf\endcsname{\let\PY@it=\textit\def\PY@tc##1{\textcolor[rgb]{0.25,0.50,0.50}{##1}}}
\expandafter\def\csname PY@tok@c1\endcsname{\let\PY@it=\textit\def\PY@tc##1{\textcolor[rgb]{0.25,0.50,0.50}{##1}}}
\expandafter\def\csname PY@tok@cs\endcsname{\let\PY@it=\textit\def\PY@tc##1{\textcolor[rgb]{0.25,0.50,0.50}{##1}}}

\def\PYZbs{\char`\\}
\def\PYZus{\char`\_}
\def\PYZob{\char`\{}
\def\PYZcb{\char`\}}
\def\PYZca{\char`\^}
\def\PYZam{\char`\&}
\def\PYZlt{\char`\<}
\def\PYZgt{\char`\>}
\def\PYZsh{\char`\#}
\def\PYZpc{\char`\%}
\def\PYZdl{\char`\$}
\def\PYZhy{\char`\-}
\def\PYZsq{\char`\'}
\def\PYZdq{\char`\"}
\def\PYZti{\char`\~}
% for compatibility with earlier versions
\def\PYZat{@}
\def\PYZlb{[}
\def\PYZrb{]}
\makeatother


    % Exact colors from NB
    \definecolor{incolor}{rgb}{0.0, 0.0, 0.5}
    \definecolor{outcolor}{rgb}{0.545, 0.0, 0.0}



    
    % Prevent overflowing lines due to hard-to-break entities
    \sloppy 
    % Setup hyperref package
    \hypersetup{
      breaklinks=true,  % so long urls are correctly broken across lines
      colorlinks=true,
      urlcolor=urlcolor,
      linkcolor=linkcolor,
      citecolor=citecolor,
      }
    % Slightly bigger margins than the latex defaults
    
    \geometry{verbose,tmargin=1in,bmargin=1in,lmargin=1in,rmargin=1in}
    
    

    \begin{document}
    
    
    \maketitle
    
    

    
    \hypertarget{numpy}{%
\section{Numpy}\label{numpy}}

Numpy is the core library for scientific computing in Python. It
provides a high-performance multidimensional array object, and tools for
working with these arrays. If you are already familiar with MATLAB, you
might find this tutorial useful to get started with Numpy.

Table of ontent * Section \ref{arrays} * Section \ref{arrays-indexing} *
Section \ref{datatypes} * Section \ref{array-math} *
Section \ref{broadcasting} * Section \ref{numpy-documentation}

    \hypertarget{arrays}{%
\subsection{Arrays }\label{arrays}}

A numpy array is a grid of values, all of the same type, and is indexed
by a tuple of nonnegative integers. The number of dimensions is the rank
of the array; the shape of an array is a tuple of integers giving the
size of the array along each dimension.

We can initialize numpy arrays from nested Python lists, and access
elements using square brackets:

    \begin{Verbatim}[commandchars=\\\{\}]
{\color{incolor}In [{\color{incolor} }]:} \PY{k+kn}{import} \PY{n+nn}{numpy} \PY{k}{as} \PY{n+nn}{np}
        
        \PY{n}{a} \PY{o}{=} \PY{n}{np}\PY{o}{.}\PY{n}{array}\PY{p}{(}\PY{p}{[}\PY{l+m+mi}{1}\PY{p}{,} \PY{l+m+mi}{2}\PY{p}{,} \PY{l+m+mi}{3}\PY{p}{]}\PY{p}{)}   \PY{c+c1}{\PYZsh{} Create a rank 1 array}
        \PY{n+nb}{print}\PY{p}{(}\PY{n+nb}{type}\PY{p}{(}\PY{n}{a}\PY{p}{)}\PY{p}{)}            \PY{c+c1}{\PYZsh{} Prints \PYZdq{}\PYZlt{}class \PYZsq{}numpy.ndarray\PYZsq{}\PYZgt{}\PYZdq{}}
        \PY{n+nb}{print}\PY{p}{(}\PY{n}{a}\PY{o}{.}\PY{n}{shape}\PY{p}{)}            \PY{c+c1}{\PYZsh{} Prints \PYZdq{}(3,)\PYZdq{}}
        \PY{n+nb}{print}\PY{p}{(}\PY{n}{a}\PY{p}{[}\PY{l+m+mi}{0}\PY{p}{]}\PY{p}{,} \PY{n}{a}\PY{p}{[}\PY{l+m+mi}{1}\PY{p}{]}\PY{p}{,} \PY{n}{a}\PY{p}{[}\PY{l+m+mi}{2}\PY{p}{]}\PY{p}{)}   \PY{c+c1}{\PYZsh{} Prints \PYZdq{}1 2 3\PYZdq{}}
        \PY{n}{a}\PY{p}{[}\PY{l+m+mi}{0}\PY{p}{]} \PY{o}{=} \PY{l+m+mi}{5}                  \PY{c+c1}{\PYZsh{} Change an element of the array}
        \PY{n+nb}{print}\PY{p}{(}\PY{n}{a}\PY{p}{)}                  \PY{c+c1}{\PYZsh{} Prints \PYZdq{}[5, 2, 3]\PYZdq{}}
        
        \PY{n}{b} \PY{o}{=} \PY{n}{np}\PY{o}{.}\PY{n}{array}\PY{p}{(}\PY{p}{[}\PY{p}{[}\PY{l+m+mi}{1}\PY{p}{,}\PY{l+m+mi}{2}\PY{p}{,}\PY{l+m+mi}{3}\PY{p}{]}\PY{p}{,}\PY{p}{[}\PY{l+m+mi}{4}\PY{p}{,}\PY{l+m+mi}{5}\PY{p}{,}\PY{l+m+mi}{6}\PY{p}{]}\PY{p}{]}\PY{p}{)}    \PY{c+c1}{\PYZsh{} Create a rank 2 array}
        \PY{n+nb}{print}\PY{p}{(}\PY{n}{b}\PY{o}{.}\PY{n}{shape}\PY{p}{)}                     \PY{c+c1}{\PYZsh{} Prints \PYZdq{}(2, 3)\PYZdq{}}
        \PY{n+nb}{print}\PY{p}{(}\PY{n}{b}\PY{p}{[}\PY{l+m+mi}{0}\PY{p}{,} \PY{l+m+mi}{0}\PY{p}{]}\PY{p}{,} \PY{n}{b}\PY{p}{[}\PY{l+m+mi}{0}\PY{p}{,} \PY{l+m+mi}{1}\PY{p}{]}\PY{p}{,} \PY{n}{b}\PY{p}{[}\PY{l+m+mi}{1}\PY{p}{,} \PY{l+m+mi}{0}\PY{p}{]}\PY{p}{)}   \PY{c+c1}{\PYZsh{} Prints \PYZdq{}1 2 4\PYZdq{}}
\end{Verbatim}


    Numpy also provides many functions to create arrays:

    \begin{Verbatim}[commandchars=\\\{\}]
{\color{incolor}In [{\color{incolor} }]:} \PY{k+kn}{import} \PY{n+nn}{numpy} \PY{k}{as} \PY{n+nn}{np}
        
        \PY{n}{a} \PY{o}{=} \PY{n}{np}\PY{o}{.}\PY{n}{zeros}\PY{p}{(}\PY{p}{(}\PY{l+m+mi}{2}\PY{p}{,}\PY{l+m+mi}{2}\PY{p}{)}\PY{p}{)}   \PY{c+c1}{\PYZsh{} Create an array of all zeros}
        \PY{n+nb}{print}\PY{p}{(}\PY{n}{a}\PY{p}{)}              \PY{c+c1}{\PYZsh{} Prints \PYZdq{}[[ 0.  0.]}
                              \PY{c+c1}{\PYZsh{}          [ 0.  0.]]\PYZdq{}}
        
        \PY{n}{b} \PY{o}{=} \PY{n}{np}\PY{o}{.}\PY{n}{ones}\PY{p}{(}\PY{p}{(}\PY{l+m+mi}{1}\PY{p}{,}\PY{l+m+mi}{2}\PY{p}{)}\PY{p}{)}    \PY{c+c1}{\PYZsh{} Create an array of all ones}
        \PY{n+nb}{print}\PY{p}{(}\PY{n}{b}\PY{p}{)}              \PY{c+c1}{\PYZsh{} Prints \PYZdq{}[[ 1.  1.]]\PYZdq{}}
        
        \PY{n}{c} \PY{o}{=} \PY{n}{np}\PY{o}{.}\PY{n}{full}\PY{p}{(}\PY{p}{(}\PY{l+m+mi}{2}\PY{p}{,}\PY{l+m+mi}{2}\PY{p}{)}\PY{p}{,} \PY{l+m+mi}{7}\PY{p}{)}  \PY{c+c1}{\PYZsh{} Create a constant array}
        \PY{n+nb}{print}\PY{p}{(}\PY{n}{c}\PY{p}{)}               \PY{c+c1}{\PYZsh{} Prints \PYZdq{}[[ 7.  7.]}
                               \PY{c+c1}{\PYZsh{}          [ 7.  7.]]\PYZdq{}}
        
        \PY{n}{d} \PY{o}{=} \PY{n}{np}\PY{o}{.}\PY{n}{eye}\PY{p}{(}\PY{l+m+mi}{2}\PY{p}{)}         \PY{c+c1}{\PYZsh{} Create a 2x2 identity matrix}
        \PY{n+nb}{print}\PY{p}{(}\PY{n}{d}\PY{p}{)}              \PY{c+c1}{\PYZsh{} Prints \PYZdq{}[[ 1.  0.]}
                              \PY{c+c1}{\PYZsh{}          [ 0.  1.]]\PYZdq{}}
        
        \PY{n}{e} \PY{o}{=} \PY{n}{np}\PY{o}{.}\PY{n}{random}\PY{o}{.}\PY{n}{random}\PY{p}{(}\PY{p}{(}\PY{l+m+mi}{2}\PY{p}{,}\PY{l+m+mi}{2}\PY{p}{)}\PY{p}{)}  \PY{c+c1}{\PYZsh{} Create an array filled with random values}
        \PY{n+nb}{print}\PY{p}{(}\PY{n}{e}\PY{p}{)}                     \PY{c+c1}{\PYZsh{} Might print \PYZdq{}[[ 0.91940167  0.08143941]}
                                     \PY{c+c1}{\PYZsh{}               [ 0.68744134  0.87236687]]\PYZdq{}}
\end{Verbatim}


    You can read about other methods of array creation in the
\href{http://docs.scipy.org/doc/numpy/reference/arrays.indexing.html}{documentation}.

    \hypertarget{arrays-indexing}{%
\subsection{Arrays indexing }\label{arrays-indexing}}

Numpy offers several ways to index into arrays.

Slicing: Similar to Python lists, numpy arrays can be sliced. Since
arrays may be multidimensional, you must specify a slice for each
dimension of the array:

    \begin{Verbatim}[commandchars=\\\{\}]
{\color{incolor}In [{\color{incolor} }]:} \PY{k+kn}{import} \PY{n+nn}{numpy} \PY{k}{as} \PY{n+nn}{np}
        
        \PY{c+c1}{\PYZsh{} Create the following rank 2 array with shape (3, 4)}
        \PY{c+c1}{\PYZsh{} [[ 1  2  3  4]}
        \PY{c+c1}{\PYZsh{}  [ 5  6  7  8]}
        \PY{c+c1}{\PYZsh{}  [ 9 10 11 12]]}
        \PY{n}{a} \PY{o}{=} \PY{n}{np}\PY{o}{.}\PY{n}{array}\PY{p}{(}\PY{p}{[}\PY{p}{[}\PY{l+m+mi}{1}\PY{p}{,}\PY{l+m+mi}{2}\PY{p}{,}\PY{l+m+mi}{3}\PY{p}{,}\PY{l+m+mi}{4}\PY{p}{]}\PY{p}{,} \PY{p}{[}\PY{l+m+mi}{5}\PY{p}{,}\PY{l+m+mi}{6}\PY{p}{,}\PY{l+m+mi}{7}\PY{p}{,}\PY{l+m+mi}{8}\PY{p}{]}\PY{p}{,} \PY{p}{[}\PY{l+m+mi}{9}\PY{p}{,}\PY{l+m+mi}{10}\PY{p}{,}\PY{l+m+mi}{11}\PY{p}{,}\PY{l+m+mi}{12}\PY{p}{]}\PY{p}{]}\PY{p}{)}
        
        \PY{c+c1}{\PYZsh{} Use slicing to pull out the subarray consisting of the first 2 rows}
        \PY{c+c1}{\PYZsh{} and columns 1 and 2; b is the following array of shape (2, 2):}
        \PY{c+c1}{\PYZsh{} [[2 3]}
        \PY{c+c1}{\PYZsh{}  [6 7]]}
        \PY{n}{b} \PY{o}{=} \PY{n}{a}\PY{p}{[}\PY{p}{:}\PY{l+m+mi}{2}\PY{p}{,} \PY{l+m+mi}{1}\PY{p}{:}\PY{l+m+mi}{3}\PY{p}{]}
        
        \PY{c+c1}{\PYZsh{} A slice of an array is a view into the same data, so modifying it}
        \PY{c+c1}{\PYZsh{} will modify the original array.}
        \PY{n+nb}{print}\PY{p}{(}\PY{n}{a}\PY{p}{[}\PY{l+m+mi}{0}\PY{p}{,} \PY{l+m+mi}{1}\PY{p}{]}\PY{p}{)}   \PY{c+c1}{\PYZsh{} Prints \PYZdq{}2\PYZdq{}}
        \PY{n}{b}\PY{p}{[}\PY{l+m+mi}{0}\PY{p}{,} \PY{l+m+mi}{0}\PY{p}{]} \PY{o}{=} \PY{l+m+mi}{77}     \PY{c+c1}{\PYZsh{} b[0, 0] is the same piece of data as a[0, 1]}
        \PY{n+nb}{print}\PY{p}{(}\PY{n}{a}\PY{p}{[}\PY{l+m+mi}{0}\PY{p}{,} \PY{l+m+mi}{1}\PY{p}{]}\PY{p}{)}   \PY{c+c1}{\PYZsh{} Prints \PYZdq{}77\PYZdq{}}
\end{Verbatim}


    You can also mix integer indexing with slice indexing. However, doing so
will yield an array of lower rank than the original array. Note that
this is quite different from the way that MATLAB handles array slicing:

    \begin{Verbatim}[commandchars=\\\{\}]
{\color{incolor}In [{\color{incolor} }]:} \PY{k+kn}{import} \PY{n+nn}{numpy} \PY{k}{as} \PY{n+nn}{np}
        
        \PY{c+c1}{\PYZsh{} Create the following rank 2 array with shape (3, 4)}
        \PY{c+c1}{\PYZsh{} [[ 1  2  3  4]}
        \PY{c+c1}{\PYZsh{}  [ 5  6  7  8]}
        \PY{c+c1}{\PYZsh{}  [ 9 10 11 12]]}
        \PY{n}{a} \PY{o}{=} \PY{n}{np}\PY{o}{.}\PY{n}{array}\PY{p}{(}\PY{p}{[}\PY{p}{[}\PY{l+m+mi}{1}\PY{p}{,}\PY{l+m+mi}{2}\PY{p}{,}\PY{l+m+mi}{3}\PY{p}{,}\PY{l+m+mi}{4}\PY{p}{]}\PY{p}{,} \PY{p}{[}\PY{l+m+mi}{5}\PY{p}{,}\PY{l+m+mi}{6}\PY{p}{,}\PY{l+m+mi}{7}\PY{p}{,}\PY{l+m+mi}{8}\PY{p}{]}\PY{p}{,} \PY{p}{[}\PY{l+m+mi}{9}\PY{p}{,}\PY{l+m+mi}{10}\PY{p}{,}\PY{l+m+mi}{11}\PY{p}{,}\PY{l+m+mi}{12}\PY{p}{]}\PY{p}{]}\PY{p}{)}
        
        \PY{c+c1}{\PYZsh{} Two ways of accessing the data in the middle row of the array.}
        \PY{c+c1}{\PYZsh{} Mixing integer indexing with slices yields an array of lower rank,}
        \PY{c+c1}{\PYZsh{} while using only slices yields an array of the same rank as the}
        \PY{c+c1}{\PYZsh{} original array:}
        \PY{n}{row\PYZus{}r1} \PY{o}{=} \PY{n}{a}\PY{p}{[}\PY{l+m+mi}{1}\PY{p}{,} \PY{p}{:}\PY{p}{]}    \PY{c+c1}{\PYZsh{} Rank 1 view of the second row of a}
        \PY{n}{row\PYZus{}r2} \PY{o}{=} \PY{n}{a}\PY{p}{[}\PY{l+m+mi}{1}\PY{p}{:}\PY{l+m+mi}{2}\PY{p}{,} \PY{p}{:}\PY{p}{]}  \PY{c+c1}{\PYZsh{} Rank 2 view of the second row of a}
        \PY{n+nb}{print}\PY{p}{(}\PY{n}{row\PYZus{}r1}\PY{p}{,} \PY{n}{row\PYZus{}r1}\PY{o}{.}\PY{n}{shape}\PY{p}{)}  \PY{c+c1}{\PYZsh{} Prints \PYZdq{}[5 6 7 8] (4,)\PYZdq{}}
        \PY{n+nb}{print}\PY{p}{(}\PY{n}{row\PYZus{}r2}\PY{p}{,} \PY{n}{row\PYZus{}r2}\PY{o}{.}\PY{n}{shape}\PY{p}{)}  \PY{c+c1}{\PYZsh{} Prints \PYZdq{}[[5 6 7 8]] (1, 4)\PYZdq{}}
        
        \PY{c+c1}{\PYZsh{} We can make the same distinction when accessing columns of an array:}
        \PY{n}{col\PYZus{}r1} \PY{o}{=} \PY{n}{a}\PY{p}{[}\PY{p}{:}\PY{p}{,} \PY{l+m+mi}{1}\PY{p}{]}
        \PY{n}{col\PYZus{}r2} \PY{o}{=} \PY{n}{a}\PY{p}{[}\PY{p}{:}\PY{p}{,} \PY{l+m+mi}{1}\PY{p}{:}\PY{l+m+mi}{2}\PY{p}{]}
        \PY{n+nb}{print}\PY{p}{(}\PY{n}{col\PYZus{}r1}\PY{p}{,} \PY{n}{col\PYZus{}r1}\PY{o}{.}\PY{n}{shape}\PY{p}{)}  \PY{c+c1}{\PYZsh{} Prints \PYZdq{}[ 2  6 10] (3,)\PYZdq{}}
        \PY{n+nb}{print}\PY{p}{(}\PY{n}{col\PYZus{}r2}\PY{p}{,} \PY{n}{col\PYZus{}r2}\PY{o}{.}\PY{n}{shape}\PY{p}{)}  \PY{c+c1}{\PYZsh{} Prints \PYZdq{}[[ 2]}
                                     \PY{c+c1}{\PYZsh{}          [ 6]}
                                     \PY{c+c1}{\PYZsh{}          [10]] (3, 1)\PYZdq{}}
\end{Verbatim}


    Integer array indexing: When you index into numpy arrays using slicing,
the resulting array view will always be a subarray of the original
array. In contrast, integer array indexing allows you to construct
arbitrary arrays using the data from another array. Here is an example:

    \begin{Verbatim}[commandchars=\\\{\}]
{\color{incolor}In [{\color{incolor} }]:} \PY{k+kn}{import} \PY{n+nn}{numpy} \PY{k}{as} \PY{n+nn}{np}
        
        \PY{n}{a} \PY{o}{=} \PY{n}{np}\PY{o}{.}\PY{n}{array}\PY{p}{(}\PY{p}{[}\PY{p}{[}\PY{l+m+mi}{1}\PY{p}{,}\PY{l+m+mi}{2}\PY{p}{]}\PY{p}{,} \PY{p}{[}\PY{l+m+mi}{3}\PY{p}{,} \PY{l+m+mi}{4}\PY{p}{]}\PY{p}{,} \PY{p}{[}\PY{l+m+mi}{5}\PY{p}{,} \PY{l+m+mi}{6}\PY{p}{]}\PY{p}{]}\PY{p}{)}
        
        \PY{c+c1}{\PYZsh{} An example of integer array indexing.}
        \PY{c+c1}{\PYZsh{} The returned array will have shape (3,) and}
        \PY{n+nb}{print}\PY{p}{(}\PY{n}{a}\PY{p}{[}\PY{p}{[}\PY{l+m+mi}{0}\PY{p}{,} \PY{l+m+mi}{1}\PY{p}{,} \PY{l+m+mi}{2}\PY{p}{]}\PY{p}{,} \PY{p}{[}\PY{l+m+mi}{0}\PY{p}{,} \PY{l+m+mi}{1}\PY{p}{,} \PY{l+m+mi}{0}\PY{p}{]}\PY{p}{]}\PY{p}{)}  \PY{c+c1}{\PYZsh{} Prints \PYZdq{}[1 4 5]\PYZdq{}}
        
        \PY{c+c1}{\PYZsh{} The above example of integer array indexing is equivalent to this:}
        \PY{n+nb}{print}\PY{p}{(}\PY{n}{np}\PY{o}{.}\PY{n}{array}\PY{p}{(}\PY{p}{[}\PY{n}{a}\PY{p}{[}\PY{l+m+mi}{0}\PY{p}{,} \PY{l+m+mi}{0}\PY{p}{]}\PY{p}{,} \PY{n}{a}\PY{p}{[}\PY{l+m+mi}{1}\PY{p}{,} \PY{l+m+mi}{1}\PY{p}{]}\PY{p}{,} \PY{n}{a}\PY{p}{[}\PY{l+m+mi}{2}\PY{p}{,} \PY{l+m+mi}{0}\PY{p}{]}\PY{p}{]}\PY{p}{)}\PY{p}{)}  \PY{c+c1}{\PYZsh{} Prints \PYZdq{}[1 4 5]\PYZdq{}}
        
        \PY{c+c1}{\PYZsh{} When using integer array indexing, you can reuse the same}
        \PY{c+c1}{\PYZsh{} element from the source array:}
        \PY{n+nb}{print}\PY{p}{(}\PY{n}{a}\PY{p}{[}\PY{p}{[}\PY{l+m+mi}{0}\PY{p}{,} \PY{l+m+mi}{0}\PY{p}{]}\PY{p}{,} \PY{p}{[}\PY{l+m+mi}{1}\PY{p}{,} \PY{l+m+mi}{1}\PY{p}{]}\PY{p}{]}\PY{p}{)}  \PY{c+c1}{\PYZsh{} Prints \PYZdq{}[2 2]\PYZdq{}}
        
        \PY{c+c1}{\PYZsh{} Equivalent to the previous integer array indexing example}
        \PY{n+nb}{print}\PY{p}{(}\PY{n}{np}\PY{o}{.}\PY{n}{array}\PY{p}{(}\PY{p}{[}\PY{n}{a}\PY{p}{[}\PY{l+m+mi}{0}\PY{p}{,} \PY{l+m+mi}{1}\PY{p}{]}\PY{p}{,} \PY{n}{a}\PY{p}{[}\PY{l+m+mi}{0}\PY{p}{,} \PY{l+m+mi}{1}\PY{p}{]}\PY{p}{]}\PY{p}{)}\PY{p}{)}  \PY{c+c1}{\PYZsh{} Prints \PYZdq{}[2 2]\PYZdq{}}
\end{Verbatim}


    One useful trick with integer array indexing is selecting or mutating
one element from each row of a matrix:

    \begin{Verbatim}[commandchars=\\\{\}]
{\color{incolor}In [{\color{incolor} }]:} \PY{k+kn}{import} \PY{n+nn}{numpy} \PY{k}{as} \PY{n+nn}{np}
        
        \PY{c+c1}{\PYZsh{} Create a new array from which we will select elements}
        \PY{n}{a} \PY{o}{=} \PY{n}{np}\PY{o}{.}\PY{n}{array}\PY{p}{(}\PY{p}{[}\PY{p}{[}\PY{l+m+mi}{1}\PY{p}{,}\PY{l+m+mi}{2}\PY{p}{,}\PY{l+m+mi}{3}\PY{p}{]}\PY{p}{,} \PY{p}{[}\PY{l+m+mi}{4}\PY{p}{,}\PY{l+m+mi}{5}\PY{p}{,}\PY{l+m+mi}{6}\PY{p}{]}\PY{p}{,} \PY{p}{[}\PY{l+m+mi}{7}\PY{p}{,}\PY{l+m+mi}{8}\PY{p}{,}\PY{l+m+mi}{9}\PY{p}{]}\PY{p}{,} \PY{p}{[}\PY{l+m+mi}{10}\PY{p}{,} \PY{l+m+mi}{11}\PY{p}{,} \PY{l+m+mi}{12}\PY{p}{]}\PY{p}{]}\PY{p}{)}
        
        \PY{n+nb}{print}\PY{p}{(}\PY{n}{a}\PY{p}{)}  \PY{c+c1}{\PYZsh{} prints \PYZdq{}array([[ 1,  2,  3],}
                  \PY{c+c1}{\PYZsh{}                [ 4,  5,  6],}
                  \PY{c+c1}{\PYZsh{}                [ 7,  8,  9],}
                  \PY{c+c1}{\PYZsh{}                [10, 11, 12]])\PYZdq{}}
        
        \PY{c+c1}{\PYZsh{} Create an array of indices}
        \PY{n}{b} \PY{o}{=} \PY{n}{np}\PY{o}{.}\PY{n}{array}\PY{p}{(}\PY{p}{[}\PY{l+m+mi}{0}\PY{p}{,} \PY{l+m+mi}{2}\PY{p}{,} \PY{l+m+mi}{0}\PY{p}{,} \PY{l+m+mi}{1}\PY{p}{]}\PY{p}{)}
        
        \PY{c+c1}{\PYZsh{} Select one element from each row of a using the indices in b}
        \PY{n+nb}{print}\PY{p}{(}\PY{n}{a}\PY{p}{[}\PY{n}{np}\PY{o}{.}\PY{n}{arange}\PY{p}{(}\PY{l+m+mi}{4}\PY{p}{)}\PY{p}{,} \PY{n}{b}\PY{p}{]}\PY{p}{)}  \PY{c+c1}{\PYZsh{} Prints \PYZdq{}[ 1  6  7 11]\PYZdq{}}
        
        \PY{c+c1}{\PYZsh{} Mutate one element from each row of a using the indices in b}
        \PY{n}{a}\PY{p}{[}\PY{n}{np}\PY{o}{.}\PY{n}{arange}\PY{p}{(}\PY{l+m+mi}{4}\PY{p}{)}\PY{p}{,} \PY{n}{b}\PY{p}{]} \PY{o}{+}\PY{o}{=} \PY{l+m+mi}{10}
        
        \PY{n+nb}{print}\PY{p}{(}\PY{n}{a}\PY{p}{)}  \PY{c+c1}{\PYZsh{} prints \PYZdq{}array([[11,  2,  3],}
                  \PY{c+c1}{\PYZsh{}                [ 4,  5, 16],}
                  \PY{c+c1}{\PYZsh{}                [17,  8,  9],}
                  \PY{c+c1}{\PYZsh{}                [10, 21, 12]])}
\end{Verbatim}


    Boolean array indexing: Boolean array indexing lets you pick out
arbitrary elements of an array. Frequently this type of indexing is used
to select the elements of an array that satisfy some condition. Here is
an example:

    \begin{Verbatim}[commandchars=\\\{\}]
{\color{incolor}In [{\color{incolor} }]:} \PY{k+kn}{import} \PY{n+nn}{numpy} \PY{k}{as} \PY{n+nn}{np}
        
        \PY{n}{a} \PY{o}{=} \PY{n}{np}\PY{o}{.}\PY{n}{array}\PY{p}{(}\PY{p}{[}\PY{p}{[}\PY{l+m+mi}{1}\PY{p}{,}\PY{l+m+mi}{2}\PY{p}{]}\PY{p}{,} \PY{p}{[}\PY{l+m+mi}{3}\PY{p}{,} \PY{l+m+mi}{4}\PY{p}{]}\PY{p}{,} \PY{p}{[}\PY{l+m+mi}{5}\PY{p}{,} \PY{l+m+mi}{6}\PY{p}{]}\PY{p}{]}\PY{p}{)}
        
        \PY{n}{bool\PYZus{}idx} \PY{o}{=} \PY{p}{(}\PY{n}{a} \PY{o}{\PYZgt{}} \PY{l+m+mi}{2}\PY{p}{)}   \PY{c+c1}{\PYZsh{} Find the elements of a that are bigger than 2;}
                             \PY{c+c1}{\PYZsh{} this returns a numpy array of Booleans of the same}
                             \PY{c+c1}{\PYZsh{} shape as a, where each slot of bool\PYZus{}idx tells}
                             \PY{c+c1}{\PYZsh{} whether that element of a is \PYZgt{} 2.}
        
        \PY{n+nb}{print}\PY{p}{(}\PY{n}{bool\PYZus{}idx}\PY{p}{)}      \PY{c+c1}{\PYZsh{} Prints \PYZdq{}[[False False]}
                             \PY{c+c1}{\PYZsh{}          [ True  True]}
                             \PY{c+c1}{\PYZsh{}          [ True  True]]\PYZdq{}}
        
        \PY{c+c1}{\PYZsh{} We use boolean array indexing to construct a rank 1 array}
        \PY{c+c1}{\PYZsh{} consisting of the elements of a corresponding to the True values}
        \PY{c+c1}{\PYZsh{} of bool\PYZus{}idx}
        \PY{n+nb}{print}\PY{p}{(}\PY{n}{a}\PY{p}{[}\PY{n}{bool\PYZus{}idx}\PY{p}{]}\PY{p}{)}  \PY{c+c1}{\PYZsh{} Prints \PYZdq{}[3 4 5 6]\PYZdq{}}
        
        \PY{c+c1}{\PYZsh{} We can do all of the above in a single concise statement:}
        \PY{n+nb}{print}\PY{p}{(}\PY{n}{a}\PY{p}{[}\PY{n}{a} \PY{o}{\PYZgt{}} \PY{l+m+mi}{2}\PY{p}{]}\PY{p}{)}     \PY{c+c1}{\PYZsh{} Prints \PYZdq{}[3 4 5 6]\PYZdq{}}
\end{Verbatim}


    For brevity we have left out a lot of details about numpy array
indexing; if you want to know more you should read the
\href{http://docs.scipy.org/doc/numpy/reference/arrays.indexing.html}{documentation}.

    \hypertarget{datatypes}{%
\subsection{Datatypes }\label{datatypes}}

Every numpy array is a grid of elements of the same type. Numpy provides
a large set of numeric datatypes that you can use to construct arrays.
Numpy tries to guess a datatype when you create an array, but functions
that construct arrays usually also include an optional argument to
explicitly specify the datatype. Here is an example:

    \begin{Verbatim}[commandchars=\\\{\}]
{\color{incolor}In [{\color{incolor} }]:} \PY{k+kn}{import} \PY{n+nn}{numpy} \PY{k}{as} \PY{n+nn}{np}
        
        \PY{n}{x} \PY{o}{=} \PY{n}{np}\PY{o}{.}\PY{n}{array}\PY{p}{(}\PY{p}{[}\PY{l+m+mi}{1}\PY{p}{,} \PY{l+m+mi}{2}\PY{p}{]}\PY{p}{)}   \PY{c+c1}{\PYZsh{} Let numpy choose the datatype}
        \PY{n+nb}{print}\PY{p}{(}\PY{n}{x}\PY{o}{.}\PY{n}{dtype}\PY{p}{)}         \PY{c+c1}{\PYZsh{} Prints \PYZdq{}int64\PYZdq{}}
        
        \PY{n}{x} \PY{o}{=} \PY{n}{np}\PY{o}{.}\PY{n}{array}\PY{p}{(}\PY{p}{[}\PY{l+m+mf}{1.0}\PY{p}{,} \PY{l+m+mf}{2.0}\PY{p}{]}\PY{p}{)}   \PY{c+c1}{\PYZsh{} Let numpy choose the datatype}
        \PY{n+nb}{print}\PY{p}{(}\PY{n}{x}\PY{o}{.}\PY{n}{dtype}\PY{p}{)}             \PY{c+c1}{\PYZsh{} Prints \PYZdq{}float64\PYZdq{}}
        
        \PY{n}{x} \PY{o}{=} \PY{n}{np}\PY{o}{.}\PY{n}{array}\PY{p}{(}\PY{p}{[}\PY{l+m+mi}{1}\PY{p}{,} \PY{l+m+mi}{2}\PY{p}{]}\PY{p}{,} \PY{n}{dtype}\PY{o}{=}\PY{n}{np}\PY{o}{.}\PY{n}{int64}\PY{p}{)}   \PY{c+c1}{\PYZsh{} Force a particular datatype}
        \PY{n+nb}{print}\PY{p}{(}\PY{n}{x}\PY{o}{.}\PY{n}{dtype}\PY{p}{)}                         \PY{c+c1}{\PYZsh{} Prints \PYZdq{}int64\PYZdq{}}
\end{Verbatim}


    You can read all about numpy datatypes in the
\href{http://docs.scipy.org/doc/numpy/reference/arrays.dtypes.html}{documentation}.

    \hypertarget{array-math}{%
\subsection{Array math }\label{array-math}}

Basic mathematical functions operate elementwise on arrays, and are
available both as operator overloads and as functions in the numpy
module:

    \begin{Verbatim}[commandchars=\\\{\}]
{\color{incolor}In [{\color{incolor} }]:} \PY{k+kn}{import} \PY{n+nn}{numpy} \PY{k}{as} \PY{n+nn}{np}
        
        \PY{n}{x} \PY{o}{=} \PY{n}{np}\PY{o}{.}\PY{n}{array}\PY{p}{(}\PY{p}{[}\PY{p}{[}\PY{l+m+mi}{1}\PY{p}{,}\PY{l+m+mi}{2}\PY{p}{]}\PY{p}{,}\PY{p}{[}\PY{l+m+mi}{3}\PY{p}{,}\PY{l+m+mi}{4}\PY{p}{]}\PY{p}{]}\PY{p}{,} \PY{n}{dtype}\PY{o}{=}\PY{n}{np}\PY{o}{.}\PY{n}{float64}\PY{p}{)}
        \PY{n}{y} \PY{o}{=} \PY{n}{np}\PY{o}{.}\PY{n}{array}\PY{p}{(}\PY{p}{[}\PY{p}{[}\PY{l+m+mi}{5}\PY{p}{,}\PY{l+m+mi}{6}\PY{p}{]}\PY{p}{,}\PY{p}{[}\PY{l+m+mi}{7}\PY{p}{,}\PY{l+m+mi}{8}\PY{p}{]}\PY{p}{]}\PY{p}{,} \PY{n}{dtype}\PY{o}{=}\PY{n}{np}\PY{o}{.}\PY{n}{float64}\PY{p}{)}
        
        \PY{c+c1}{\PYZsh{} Elementwise sum; both produce the array}
        \PY{c+c1}{\PYZsh{} [[ 6.0  8.0]}
        \PY{c+c1}{\PYZsh{}  [10.0 12.0]]}
        \PY{n+nb}{print}\PY{p}{(}\PY{n}{x} \PY{o}{+} \PY{n}{y}\PY{p}{)}
        \PY{n+nb}{print}\PY{p}{(}\PY{n}{np}\PY{o}{.}\PY{n}{add}\PY{p}{(}\PY{n}{x}\PY{p}{,} \PY{n}{y}\PY{p}{)}\PY{p}{)}
        
        \PY{c+c1}{\PYZsh{} Elementwise difference; both produce the array}
        \PY{c+c1}{\PYZsh{} [[\PYZhy{}4.0 \PYZhy{}4.0]}
        \PY{c+c1}{\PYZsh{}  [\PYZhy{}4.0 \PYZhy{}4.0]]}
        \PY{n+nb}{print}\PY{p}{(}\PY{n}{x} \PY{o}{\PYZhy{}} \PY{n}{y}\PY{p}{)}
        \PY{n+nb}{print}\PY{p}{(}\PY{n}{np}\PY{o}{.}\PY{n}{subtract}\PY{p}{(}\PY{n}{x}\PY{p}{,} \PY{n}{y}\PY{p}{)}\PY{p}{)}
        
        \PY{c+c1}{\PYZsh{} Elementwise product; both produce the array}
        \PY{c+c1}{\PYZsh{} [[ 5.0 12.0]}
        \PY{c+c1}{\PYZsh{}  [21.0 32.0]]}
        \PY{n+nb}{print}\PY{p}{(}\PY{n}{x} \PY{o}{*} \PY{n}{y}\PY{p}{)}
        \PY{n+nb}{print}\PY{p}{(}\PY{n}{np}\PY{o}{.}\PY{n}{multiply}\PY{p}{(}\PY{n}{x}\PY{p}{,} \PY{n}{y}\PY{p}{)}\PY{p}{)}
        
        \PY{c+c1}{\PYZsh{} Elementwise division; both produce the array}
        \PY{c+c1}{\PYZsh{} [[ 0.2         0.33333333]}
        \PY{c+c1}{\PYZsh{}  [ 0.42857143  0.5       ]]}
        \PY{n+nb}{print}\PY{p}{(}\PY{n}{x} \PY{o}{/} \PY{n}{y}\PY{p}{)}
        \PY{n+nb}{print}\PY{p}{(}\PY{n}{np}\PY{o}{.}\PY{n}{divide}\PY{p}{(}\PY{n}{x}\PY{p}{,} \PY{n}{y}\PY{p}{)}\PY{p}{)}
        
        \PY{c+c1}{\PYZsh{} Elementwise square root; produces the array}
        \PY{c+c1}{\PYZsh{} [[ 1.          1.41421356]}
        \PY{c+c1}{\PYZsh{}  [ 1.73205081  2.        ]]}
        \PY{n+nb}{print}\PY{p}{(}\PY{n}{np}\PY{o}{.}\PY{n}{sqrt}\PY{p}{(}\PY{n}{x}\PY{p}{)}\PY{p}{)}
\end{Verbatim}


    Note that unlike MATLAB, * is elementwise multiplication, not matrix
multiplication. We instead use the dot function to compute inner
products of vectors, to multiply a vector by a matrix, and to multiply
matrices. dot is available both as a function in the numpy module and as
an instance method of array objects:

    \begin{Verbatim}[commandchars=\\\{\}]
{\color{incolor}In [{\color{incolor} }]:} \PY{k+kn}{import} \PY{n+nn}{numpy} \PY{k}{as} \PY{n+nn}{np}
        
        \PY{n}{x} \PY{o}{=} \PY{n}{np}\PY{o}{.}\PY{n}{array}\PY{p}{(}\PY{p}{[}\PY{p}{[}\PY{l+m+mi}{1}\PY{p}{,}\PY{l+m+mi}{2}\PY{p}{]}\PY{p}{,}\PY{p}{[}\PY{l+m+mi}{3}\PY{p}{,}\PY{l+m+mi}{4}\PY{p}{]}\PY{p}{]}\PY{p}{)}
        \PY{n}{y} \PY{o}{=} \PY{n}{np}\PY{o}{.}\PY{n}{array}\PY{p}{(}\PY{p}{[}\PY{p}{[}\PY{l+m+mi}{5}\PY{p}{,}\PY{l+m+mi}{6}\PY{p}{]}\PY{p}{,}\PY{p}{[}\PY{l+m+mi}{7}\PY{p}{,}\PY{l+m+mi}{8}\PY{p}{]}\PY{p}{]}\PY{p}{)}
        
        \PY{n}{v} \PY{o}{=} \PY{n}{np}\PY{o}{.}\PY{n}{array}\PY{p}{(}\PY{p}{[}\PY{l+m+mi}{9}\PY{p}{,}\PY{l+m+mi}{10}\PY{p}{]}\PY{p}{)}
        \PY{n}{w} \PY{o}{=} \PY{n}{np}\PY{o}{.}\PY{n}{array}\PY{p}{(}\PY{p}{[}\PY{l+m+mi}{11}\PY{p}{,} \PY{l+m+mi}{12}\PY{p}{]}\PY{p}{)}
        
        \PY{c+c1}{\PYZsh{} Inner product of vectors; both produce 219}
        \PY{n+nb}{print}\PY{p}{(}\PY{n}{v}\PY{o}{.}\PY{n}{dot}\PY{p}{(}\PY{n}{w}\PY{p}{)}\PY{p}{)}
        \PY{n+nb}{print}\PY{p}{(}\PY{n}{np}\PY{o}{.}\PY{n}{dot}\PY{p}{(}\PY{n}{v}\PY{p}{,} \PY{n}{w}\PY{p}{)}\PY{p}{)}
        
        \PY{c+c1}{\PYZsh{} Matrix / vector product; both produce the rank 1 array [29 67]}
        \PY{n+nb}{print}\PY{p}{(}\PY{n}{x}\PY{o}{.}\PY{n}{dot}\PY{p}{(}\PY{n}{v}\PY{p}{)}\PY{p}{)}
        \PY{n+nb}{print}\PY{p}{(}\PY{n}{np}\PY{o}{.}\PY{n}{dot}\PY{p}{(}\PY{n}{x}\PY{p}{,} \PY{n}{v}\PY{p}{)}\PY{p}{)}
        
        \PY{c+c1}{\PYZsh{} Matrix / matrix product; both produce the rank 2 array}
        \PY{c+c1}{\PYZsh{} [[19 22]}
        \PY{c+c1}{\PYZsh{}  [43 50]]}
        \PY{n+nb}{print}\PY{p}{(}\PY{n}{x}\PY{o}{.}\PY{n}{dot}\PY{p}{(}\PY{n}{y}\PY{p}{)}\PY{p}{)}
        \PY{n+nb}{print}\PY{p}{(}\PY{n}{np}\PY{o}{.}\PY{n}{dot}\PY{p}{(}\PY{n}{x}\PY{p}{,} \PY{n}{y}\PY{p}{)}\PY{p}{)}
\end{Verbatim}


    Numpy provides many useful functions for performing computations on
arrays; one of the most useful is sum:

    \begin{Verbatim}[commandchars=\\\{\}]
{\color{incolor}In [{\color{incolor} }]:} \PY{k+kn}{import} \PY{n+nn}{numpy} \PY{k}{as} \PY{n+nn}{np}
        
        \PY{n}{x} \PY{o}{=} \PY{n}{np}\PY{o}{.}\PY{n}{array}\PY{p}{(}\PY{p}{[}\PY{p}{[}\PY{l+m+mi}{1}\PY{p}{,}\PY{l+m+mi}{2}\PY{p}{]}\PY{p}{,}\PY{p}{[}\PY{l+m+mi}{3}\PY{p}{,}\PY{l+m+mi}{4}\PY{p}{]}\PY{p}{]}\PY{p}{)}
        
        \PY{n+nb}{print}\PY{p}{(}\PY{n}{np}\PY{o}{.}\PY{n}{sum}\PY{p}{(}\PY{n}{x}\PY{p}{)}\PY{p}{)}  \PY{c+c1}{\PYZsh{} Compute sum of all elements; prints \PYZdq{}10\PYZdq{}}
        \PY{n+nb}{print}\PY{p}{(}\PY{n}{np}\PY{o}{.}\PY{n}{sum}\PY{p}{(}\PY{n}{x}\PY{p}{,} \PY{n}{axis}\PY{o}{=}\PY{l+m+mi}{0}\PY{p}{)}\PY{p}{)}  \PY{c+c1}{\PYZsh{} Compute sum of each column; prints \PYZdq{}[4 6]\PYZdq{}}
        \PY{n+nb}{print}\PY{p}{(}\PY{n}{np}\PY{o}{.}\PY{n}{sum}\PY{p}{(}\PY{n}{x}\PY{p}{,} \PY{n}{axis}\PY{o}{=}\PY{l+m+mi}{1}\PY{p}{)}\PY{p}{)}  \PY{c+c1}{\PYZsh{} Compute sum of each row; prints \PYZdq{}[3 7]\PYZdq{}}
\end{Verbatim}


    You can find the full list of mathematical functions provided by numpy
in the
\href{http://docs.scipy.org/doc/numpy/reference/routines.math.html}{documentation}.

Apart from computing mathematical functions using arrays, we frequently
need to reshape or otherwise manipulate data in arrays. The simplest
example of this type of operation is transposing a matrix; to transpose
a matrix, simply use the T attribute of an array object:

    \begin{Verbatim}[commandchars=\\\{\}]
{\color{incolor}In [{\color{incolor} }]:} \PY{k+kn}{import} \PY{n+nn}{numpy} \PY{k}{as} \PY{n+nn}{np}
        
        \PY{n}{x} \PY{o}{=} \PY{n}{np}\PY{o}{.}\PY{n}{array}\PY{p}{(}\PY{p}{[}\PY{p}{[}\PY{l+m+mi}{1}\PY{p}{,}\PY{l+m+mi}{2}\PY{p}{]}\PY{p}{,} \PY{p}{[}\PY{l+m+mi}{3}\PY{p}{,}\PY{l+m+mi}{4}\PY{p}{]}\PY{p}{]}\PY{p}{)}
        \PY{n+nb}{print}\PY{p}{(}\PY{n}{x}\PY{p}{)}    \PY{c+c1}{\PYZsh{} Prints \PYZdq{}[[1 2]}
                    \PY{c+c1}{\PYZsh{}          [3 4]]\PYZdq{}}
        \PY{n+nb}{print}\PY{p}{(}\PY{n}{x}\PY{o}{.}\PY{n}{T}\PY{p}{)}  \PY{c+c1}{\PYZsh{} Prints \PYZdq{}[[1 3]}
                    \PY{c+c1}{\PYZsh{}          [2 4]]\PYZdq{}}
        
        \PY{c+c1}{\PYZsh{} Note that taking the transpose of a rank 1 array does nothing:}
        \PY{n}{v} \PY{o}{=} \PY{n}{np}\PY{o}{.}\PY{n}{array}\PY{p}{(}\PY{p}{[}\PY{l+m+mi}{1}\PY{p}{,}\PY{l+m+mi}{2}\PY{p}{,}\PY{l+m+mi}{3}\PY{p}{]}\PY{p}{)}
        \PY{n+nb}{print}\PY{p}{(}\PY{n}{v}\PY{p}{)}    \PY{c+c1}{\PYZsh{} Prints \PYZdq{}[1 2 3]\PYZdq{}}
        \PY{n+nb}{print}\PY{p}{(}\PY{n}{v}\PY{o}{.}\PY{n}{T}\PY{p}{)}  \PY{c+c1}{\PYZsh{} Prints \PYZdq{}[1 2 3]\PYZdq{}}
\end{Verbatim}


    Numpy provides many more functions for manipulating arrays; you can see
the full list in the
\href{http://docs.scipy.org/doc/numpy/reference/routines.array-manipulation.html}{documentation}.

    \hypertarget{broadcasting}{%
\subsection{Broadcasting }\label{broadcasting}}

Broadcasting is a powerful mechanism that allows numpy to work with
arrays of different shapes when performing arithmetic operations.
Frequently we have a smaller array and a larger array, and we want to
use the smaller array multiple times to perform some operation on the
larger array.

For example, suppose that we want to add a constant vector to each row
of a matrix. We could do it like this:

    \begin{Verbatim}[commandchars=\\\{\}]
{\color{incolor}In [{\color{incolor} }]:} \PY{k+kn}{import} \PY{n+nn}{numpy} \PY{k}{as} \PY{n+nn}{np}
        
        \PY{c+c1}{\PYZsh{} We will add the vector v to each row of the matrix x,}
        \PY{c+c1}{\PYZsh{} storing the result in the matrix y}
        \PY{n}{x} \PY{o}{=} \PY{n}{np}\PY{o}{.}\PY{n}{array}\PY{p}{(}\PY{p}{[}\PY{p}{[}\PY{l+m+mi}{1}\PY{p}{,}\PY{l+m+mi}{2}\PY{p}{,}\PY{l+m+mi}{3}\PY{p}{]}\PY{p}{,} \PY{p}{[}\PY{l+m+mi}{4}\PY{p}{,}\PY{l+m+mi}{5}\PY{p}{,}\PY{l+m+mi}{6}\PY{p}{]}\PY{p}{,} \PY{p}{[}\PY{l+m+mi}{7}\PY{p}{,}\PY{l+m+mi}{8}\PY{p}{,}\PY{l+m+mi}{9}\PY{p}{]}\PY{p}{,} \PY{p}{[}\PY{l+m+mi}{10}\PY{p}{,} \PY{l+m+mi}{11}\PY{p}{,} \PY{l+m+mi}{12}\PY{p}{]}\PY{p}{]}\PY{p}{)}
        \PY{n}{v} \PY{o}{=} \PY{n}{np}\PY{o}{.}\PY{n}{array}\PY{p}{(}\PY{p}{[}\PY{l+m+mi}{1}\PY{p}{,} \PY{l+m+mi}{0}\PY{p}{,} \PY{l+m+mi}{1}\PY{p}{]}\PY{p}{)}
        \PY{n}{y} \PY{o}{=} \PY{n}{np}\PY{o}{.}\PY{n}{empty\PYZus{}like}\PY{p}{(}\PY{n}{x}\PY{p}{)}   \PY{c+c1}{\PYZsh{} Create an empty matrix with the same shape as x}
        
        \PY{c+c1}{\PYZsh{} Add the vector v to each row of the matrix x with an explicit loop}
        \PY{k}{for} \PY{n}{i} \PY{o+ow}{in} \PY{n+nb}{range}\PY{p}{(}\PY{l+m+mi}{4}\PY{p}{)}\PY{p}{:}
            \PY{n}{y}\PY{p}{[}\PY{n}{i}\PY{p}{,} \PY{p}{:}\PY{p}{]} \PY{o}{=} \PY{n}{x}\PY{p}{[}\PY{n}{i}\PY{p}{,} \PY{p}{:}\PY{p}{]} \PY{o}{+} \PY{n}{v}
        
        \PY{c+c1}{\PYZsh{} Now y is the following}
        \PY{c+c1}{\PYZsh{} [[ 2  2  4]}
        \PY{c+c1}{\PYZsh{}  [ 5  5  7]}
        \PY{c+c1}{\PYZsh{}  [ 8  8 10]}
        \PY{c+c1}{\PYZsh{}  [11 11 13]]}
        \PY{n+nb}{print}\PY{p}{(}\PY{n}{y}\PY{p}{)}
\end{Verbatim}


    This works; however when the matrix x is very large, computing an
explicit loop in Python could be slow. Note that adding the vector v to
each row of the matrix x is equivalent to forming a matrix vv by
stacking multiple copies of v vertically, then performing elementwise
summation of x and vv. We could implement this approach like this:

    \begin{Verbatim}[commandchars=\\\{\}]
{\color{incolor}In [{\color{incolor} }]:} \PY{k+kn}{import} \PY{n+nn}{numpy} \PY{k}{as} \PY{n+nn}{np}
        
        \PY{c+c1}{\PYZsh{} We will add the vector v to each row of the matrix x,}
        \PY{c+c1}{\PYZsh{} storing the result in the matrix y}
        \PY{n}{x} \PY{o}{=} \PY{n}{np}\PY{o}{.}\PY{n}{array}\PY{p}{(}\PY{p}{[}\PY{p}{[}\PY{l+m+mi}{1}\PY{p}{,}\PY{l+m+mi}{2}\PY{p}{,}\PY{l+m+mi}{3}\PY{p}{]}\PY{p}{,} \PY{p}{[}\PY{l+m+mi}{4}\PY{p}{,}\PY{l+m+mi}{5}\PY{p}{,}\PY{l+m+mi}{6}\PY{p}{]}\PY{p}{,} \PY{p}{[}\PY{l+m+mi}{7}\PY{p}{,}\PY{l+m+mi}{8}\PY{p}{,}\PY{l+m+mi}{9}\PY{p}{]}\PY{p}{,} \PY{p}{[}\PY{l+m+mi}{10}\PY{p}{,} \PY{l+m+mi}{11}\PY{p}{,} \PY{l+m+mi}{12}\PY{p}{]}\PY{p}{]}\PY{p}{)}
        \PY{n}{v} \PY{o}{=} \PY{n}{np}\PY{o}{.}\PY{n}{array}\PY{p}{(}\PY{p}{[}\PY{l+m+mi}{1}\PY{p}{,} \PY{l+m+mi}{0}\PY{p}{,} \PY{l+m+mi}{1}\PY{p}{]}\PY{p}{)}
        \PY{n}{vv} \PY{o}{=} \PY{n}{np}\PY{o}{.}\PY{n}{tile}\PY{p}{(}\PY{n}{v}\PY{p}{,} \PY{p}{(}\PY{l+m+mi}{4}\PY{p}{,} \PY{l+m+mi}{1}\PY{p}{)}\PY{p}{)}   \PY{c+c1}{\PYZsh{} Stack 4 copies of v on top of each other}
        \PY{n+nb}{print}\PY{p}{(}\PY{n}{vv}\PY{p}{)}                 \PY{c+c1}{\PYZsh{} Prints \PYZdq{}[[1 0 1]}
                                  \PY{c+c1}{\PYZsh{}          [1 0 1]}
                                  \PY{c+c1}{\PYZsh{}          [1 0 1]}
                                  \PY{c+c1}{\PYZsh{}          [1 0 1]]\PYZdq{}}
        \PY{n}{y} \PY{o}{=} \PY{n}{x} \PY{o}{+} \PY{n}{vv}  \PY{c+c1}{\PYZsh{} Add x and vv elementwise}
        \PY{n+nb}{print}\PY{p}{(}\PY{n}{y}\PY{p}{)}  \PY{c+c1}{\PYZsh{} Prints \PYZdq{}[[ 2  2  4}
                  \PY{c+c1}{\PYZsh{}          [ 5  5  7]}
                  \PY{c+c1}{\PYZsh{}          [ 8  8 10]}
                  \PY{c+c1}{\PYZsh{}          [11 11 13]]\PYZdq{}}
\end{Verbatim}


    Numpy broadcasting allows us to perform this computation without
actually creating multiple copies of v. Consider this version, using
broadcasting:

    \begin{Verbatim}[commandchars=\\\{\}]
{\color{incolor}In [{\color{incolor} }]:} \PY{k+kn}{import} \PY{n+nn}{numpy} \PY{k}{as} \PY{n+nn}{np}
        
        \PY{c+c1}{\PYZsh{} We will add the vector v to each row of the matrix x,}
        \PY{c+c1}{\PYZsh{} storing the result in the matrix y}
        \PY{n}{x} \PY{o}{=} \PY{n}{np}\PY{o}{.}\PY{n}{array}\PY{p}{(}\PY{p}{[}\PY{p}{[}\PY{l+m+mi}{1}\PY{p}{,}\PY{l+m+mi}{2}\PY{p}{,}\PY{l+m+mi}{3}\PY{p}{]}\PY{p}{,} \PY{p}{[}\PY{l+m+mi}{4}\PY{p}{,}\PY{l+m+mi}{5}\PY{p}{,}\PY{l+m+mi}{6}\PY{p}{]}\PY{p}{,} \PY{p}{[}\PY{l+m+mi}{7}\PY{p}{,}\PY{l+m+mi}{8}\PY{p}{,}\PY{l+m+mi}{9}\PY{p}{]}\PY{p}{,} \PY{p}{[}\PY{l+m+mi}{10}\PY{p}{,} \PY{l+m+mi}{11}\PY{p}{,} \PY{l+m+mi}{12}\PY{p}{]}\PY{p}{]}\PY{p}{)}
        \PY{n}{v} \PY{o}{=} \PY{n}{np}\PY{o}{.}\PY{n}{array}\PY{p}{(}\PY{p}{[}\PY{l+m+mi}{1}\PY{p}{,} \PY{l+m+mi}{0}\PY{p}{,} \PY{l+m+mi}{1}\PY{p}{]}\PY{p}{)}
        \PY{n}{y} \PY{o}{=} \PY{n}{x} \PY{o}{+} \PY{n}{v}  \PY{c+c1}{\PYZsh{} Add v to each row of x using broadcasting}
        \PY{n+nb}{print}\PY{p}{(}\PY{n}{y}\PY{p}{)}  \PY{c+c1}{\PYZsh{} Prints \PYZdq{}[[ 2  2  4]}
                  \PY{c+c1}{\PYZsh{}          [ 5  5  7]}
                  \PY{c+c1}{\PYZsh{}          [ 8  8 10]}
                  \PY{c+c1}{\PYZsh{}          [11 11 13]]\PYZdq{}}
\end{Verbatim}


    The line y = x + v works even though x has shape (4, 3) and v has shape
(3,) due to broadcasting; this line works as if v actually had shape (4,
3), where each row was a copy of v, and the sum was performed
elementwise.

Broadcasting two arrays together follows these rules:

If the arrays do not have the same rank, prepend the shape of the lower
rank array with 1s until both shapes have the same length. The two
arrays are said to be compatible in a dimension if they have the same
size in the dimension, or if one of the arrays has size 1 in that
dimension. The arrays can be broadcast together if they are compatible
in all dimensions. After broadcasting, each array behaves as if it had
shape equal to the elementwise maximum of shapes of the two input
arrays. In any dimension where one array had size 1 and the other array
had size greater than 1, the first array behaves as if it were copied
along that dimension If this explanation does not make sense, try
reading the explanation from the
\href{http://docs.scipy.org/doc/numpy/user/basics.broadcasting.html}{documentation}.

Functions that support broadcasting are known as universal functions.
You can find the list of all universal functions in the
\href{http://docs.scipy.org/doc/numpy/reference/ufuncs.html\#available-ufuncs}{documentation}.

Here are some applications of broadcasting:

    \begin{Verbatim}[commandchars=\\\{\}]
{\color{incolor}In [{\color{incolor} }]:} \PY{k+kn}{import} \PY{n+nn}{numpy} \PY{k}{as} \PY{n+nn}{np}
        
        \PY{c+c1}{\PYZsh{} Compute outer product of vectors}
        \PY{n}{v} \PY{o}{=} \PY{n}{np}\PY{o}{.}\PY{n}{array}\PY{p}{(}\PY{p}{[}\PY{l+m+mi}{1}\PY{p}{,}\PY{l+m+mi}{2}\PY{p}{,}\PY{l+m+mi}{3}\PY{p}{]}\PY{p}{)}  \PY{c+c1}{\PYZsh{} v has shape (3,)}
        \PY{n}{w} \PY{o}{=} \PY{n}{np}\PY{o}{.}\PY{n}{array}\PY{p}{(}\PY{p}{[}\PY{l+m+mi}{4}\PY{p}{,}\PY{l+m+mi}{5}\PY{p}{]}\PY{p}{)}    \PY{c+c1}{\PYZsh{} w has shape (2,)}
        \PY{c+c1}{\PYZsh{} To compute an outer product, we first reshape v to be a column}
        \PY{c+c1}{\PYZsh{} vector of shape (3, 1); we can then broadcast it against w to yield}
        \PY{c+c1}{\PYZsh{} an output of shape (3, 2), which is the outer product of v and w:}
        \PY{c+c1}{\PYZsh{} [[ 4  5]}
        \PY{c+c1}{\PYZsh{}  [ 8 10]}
        \PY{c+c1}{\PYZsh{}  [12 15]]}
        \PY{n+nb}{print}\PY{p}{(}\PY{n}{np}\PY{o}{.}\PY{n}{reshape}\PY{p}{(}\PY{n}{v}\PY{p}{,} \PY{p}{(}\PY{l+m+mi}{3}\PY{p}{,} \PY{l+m+mi}{1}\PY{p}{)}\PY{p}{)} \PY{o}{*} \PY{n}{w}\PY{p}{)}
        
        \PY{c+c1}{\PYZsh{} Add a vector to each row of a matrix}
        \PY{n}{x} \PY{o}{=} \PY{n}{np}\PY{o}{.}\PY{n}{array}\PY{p}{(}\PY{p}{[}\PY{p}{[}\PY{l+m+mi}{1}\PY{p}{,}\PY{l+m+mi}{2}\PY{p}{,}\PY{l+m+mi}{3}\PY{p}{]}\PY{p}{,} \PY{p}{[}\PY{l+m+mi}{4}\PY{p}{,}\PY{l+m+mi}{5}\PY{p}{,}\PY{l+m+mi}{6}\PY{p}{]}\PY{p}{]}\PY{p}{)}
        \PY{c+c1}{\PYZsh{} x has shape (2, 3) and v has shape (3,) so they broadcast to (2, 3),}
        \PY{c+c1}{\PYZsh{} giving the following matrix:}
        \PY{c+c1}{\PYZsh{} [[2 4 6]}
        \PY{c+c1}{\PYZsh{}  [5 7 9]]}
        \PY{n+nb}{print}\PY{p}{(}\PY{n}{x} \PY{o}{+} \PY{n}{v}\PY{p}{)}
        
        \PY{c+c1}{\PYZsh{} Add a vector to each column of a matrix}
        \PY{c+c1}{\PYZsh{} x has shape (2, 3) and w has shape (2,).}
        \PY{c+c1}{\PYZsh{} If we transpose x then it has shape (3, 2) and can be broadcast}
        \PY{c+c1}{\PYZsh{} against w to yield a result of shape (3, 2); transposing this result}
        \PY{c+c1}{\PYZsh{} yields the final result of shape (2, 3) which is the matrix x with}
        \PY{c+c1}{\PYZsh{} the vector w added to each column. Gives the following matrix:}
        \PY{c+c1}{\PYZsh{} [[ 5  6  7]}
        \PY{c+c1}{\PYZsh{}  [ 9 10 11]]}
        \PY{n+nb}{print}\PY{p}{(}\PY{p}{(}\PY{n}{x}\PY{o}{.}\PY{n}{T} \PY{o}{+} \PY{n}{w}\PY{p}{)}\PY{o}{.}\PY{n}{T}\PY{p}{)}
        \PY{c+c1}{\PYZsh{} Another solution is to reshape w to be a column vector of shape (2, 1);}
        \PY{c+c1}{\PYZsh{} we can then broadcast it directly against x to produce the same}
        \PY{c+c1}{\PYZsh{} output.}
        \PY{n+nb}{print}\PY{p}{(}\PY{n}{x} \PY{o}{+} \PY{n}{np}\PY{o}{.}\PY{n}{reshape}\PY{p}{(}\PY{n}{w}\PY{p}{,} \PY{p}{(}\PY{l+m+mi}{2}\PY{p}{,} \PY{l+m+mi}{1}\PY{p}{)}\PY{p}{)}\PY{p}{)}
        
        \PY{c+c1}{\PYZsh{} Multiply a matrix by a constant:}
        \PY{c+c1}{\PYZsh{} x has shape (2, 3). Numpy treats scalars as arrays of shape ();}
        \PY{c+c1}{\PYZsh{} these can be broadcast together to shape (2, 3), producing the}
        \PY{c+c1}{\PYZsh{} following array:}
        \PY{c+c1}{\PYZsh{} [[ 2  4  6]}
        \PY{c+c1}{\PYZsh{}  [ 8 10 12]]}
        \PY{n+nb}{print}\PY{p}{(}\PY{n}{x} \PY{o}{*} \PY{l+m+mi}{2}\PY{p}{)}
\end{Verbatim}


    Broadcasting typically makes your code more concise and faster, so you
should strive to use it where possible.

    \hypertarget{numpy-documentation}{%
\subsection{Numpy Documentation }\label{numpy-documentation}}

This brief overview has touched on many of the important things that you
need to know about numpy, but is far from complete. Check out the numpy
reference to find out much more about numpy.


    % Add a bibliography block to the postdoc
    
    
    
    \end{document}
